\chapter{Introduction}

The idea presented in this paper is rather simple, but it requires somewhat
sophisticated background knowledge. This chapter serves to provide said
background knowledge. It is assumed the reader is familiar with the basic
concepts in computer programming, and has at least some experience in an
imperative programming language, such as C, C++, Python, or Java.

\section{Functional languages}

Functional programming languages are programming languages that treat functions
as first-class values. In most programming languages, there's a rather sharp
distinction between data, and code that modifies the data. In functional
programming languages, that distinction is somewhat blurred.

\begin{example}[Factorial function]
  Recall the familiar factorial function in mathematics:

  \begin{rclmath}
    ! & : & \N \to \N \\
    0! & = & 1 \\
    n! & = & n \cdot (n - 1)! \\
  \end{rclmath}
  
  \lstinputlisting[caption={Factorial function in C}, language=C, firstline=3, lastline=14]{code/factorial.c}

  In Idris, a functional language, this is the factorial function:

  \lstinputlisting[caption={Factorial function in Idris}, firstline=3, lastline=5]{code/factorial.idr}
\end{example}